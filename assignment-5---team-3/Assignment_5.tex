\documentclass[paper=letter, fontsize=11pt]{scrartcl} % A4 paper and 11pt font size

\usepackage{enumitem}
\usepackage{listings}
\usepackage[T1]{fontenc} % Use 8-bit encoding that has 256 glyphs
\usepackage{fourier} % Use the Adobe Utopia font for the document - comment this line to return to the LaTeX default
\usepackage[english]{babel} % English language/hyphenation
\usepackage{amsmath,amsfonts,amsthm} % Math packages
\usepackage{lipsum} % Used for inserting dummy 'Lorem ipsum' text into the template
\usepackage{sectsty} % Allows customizing section commands
\allsectionsfont{\centering \normalfont\scshape} % Make all sections centered, the default font and small caps
\usepackage{fancyhdr} % Custom headers and footers
\pagestyle{fancyplain} % Makes all pages in the document conform to the custom headers and footers
\fancyhead{} % No page header - if you want one, create it in the same way as the footers below
\fancyfoot[L]{} % Empty left footer
\fancyfoot[C]{} % Empty center footer
% \fancyfoot[R]{\thepage} % Page numbering for right footer
\renewcommand{\headrulewidth}{0pt} % Remove header underlines
\renewcommand{\footrulewidth}{0pt} % Remove footer underlines
\setlength{\headheight}{13.6pt} % Customize the height of the header

\setlength\parindent{0pt} % Removes all indentation from paragraphs - comment this line for an assignment with lots of text

\usepackage[margin=0.75in]{geometry}
\usepackage{hyperref}
%----------------------------------------------------------------------------------------
%   TITLE SECTION
%----------------------------------------------------------------------------------------

\newcommand{\horrule}[1]{\rule{\linewidth}{#1}} % Create horizontal rule command with 1 argument of height

\title{ 
    \normalfont \normalsize 
    \textsc{CS 3540 Intro. to Software Engineering, Spring 2015} \\ [25pt] % Your university, school and/or department name(s)
    \horrule{0.5pt} \\[0.4cm] % Thin top horizontal rule
    \huge Assignment \#5    \\ % The assignment title
    \horrule{2pt} \\[0.5cm] % Thick bottom horizontal rule
}

% \author{John Smith} % Your name

% \date{\normalsize\today} % Today's date or a custom date

\begin{document}

    \begin{center}
         Assignment \#5\\
        \small CS 3540 Intro. to Software Engineering, Spring 2015 \\
        \small Dr. Robert Green \\
        \huge Test Driven Development
    \end{center}
    
    \textbf{Due Date:} April 14 @ 5.00 pm (Pull Request must be generated by this time). \\

    This assignment is intended to familiarize you with Test Driven Development by implementing the User story ``Preorder your coffee with a gift card'' from Chapter 8 using Test-Driven Development in the language of your choice (Python preferred). Your implementation must include the following core objects: \texttt{OrderInformation}, \texttt{GiftCard}, and \texttt{Receipt}. You must also create the appropriate testing code. Store all code for this project in Stash using Git. \\

    This assignment must be completed using test-driven development. In other words, tests must be written before developing objects. In order to ensure that this occurs, tests must be committed using Git before any code is written. Also, each team member must contribute tests as well as actual code. There must be at least 5 tests for each object.\\
    
    \textbf{Rubric:} 55 points total as listed below. 
    
    \begin{itemize}[noitemsep]
        \item \textbf{(15 Points)} Implementation must include the following core classes: \texttt{OrderInformation}, \texttt{GiftCard}, \\ and \texttt{Receipt}.
        \item \textbf{(15 Points)} Test-Driven Development must be used including a minimum of 5 Tests per object. This means that you must develop meaningful tests to test the functionality of each test. \textit{All tests must be in the same file and/or class.}
        \item \textbf{(5 Points)} Fork the repository ``Assignment 5'' under the project ``Assignment \#5 -- Test-Driven Development''.
        \item \textbf{(5 Points)} Forked repository must be named ``Assignment 5 -- Team XX'' and must be under the project ``Assignment \#5 -- Test Driven Development''. 
        \item \textbf{(5 Points)} Each student must create a branch for their own work using their username as the branch name. 

        \item \textbf{(10 Points)} All branches must be merged into the main branch before turning in the assignment. While I have covered the basics of merging, I expect you to look into this topic and come to me outside of class with any questions.

        \item You can self sign-up for your groups via Canvas under People --> ``Assignment \#5 Groups''. Groups must be 2 to 4 people and be created by April 2 at 5 pm. There are no exceptions. Failure to meet this requirement will result in a grade of 0 on this assignment.
    \end{itemize}

    \textbf{Additional Requirements}
    \begin{itemize}[noitemsep]
        \item Groups must be between 2 and 4 members, no exceptions.
        \item Each team member must contribute both code and tests.
        \item Each student must work on their own branch.
        \item Chelsea Zhang must be assigned as the reviewer when creating your pull request.
        \item Assignment must be turned in via Pull Request.
        \item Your code should look professional! Treat this assignment as if you were handing it to a real-world client.  You must produce code that is well-formatted, well-aligned, and easy to read.
        \item \textit{I strongly suggest you use command line only tools for using Git!}
    \end{itemize}

    % \textbf{Detailed Instructions:} For this assignment, you are to implement the User story ``Preorder your coffee with a gift card'' from Chapter 8 using Test-Driven Development in the language of your choice (Python preferred). Your implementation must include the following core objects: OrderInformation, GiftCard, Receipt, OrderProcessor, and DBAccessor (Interface). You must also create the appropriate testing code. Store all code for this project in Stash using Git. \\

    All work is to be version controlled using Stash on voyager (http://voyager.cs.bgsu.edu:7990/) under the project ``Assignment \#5 -- Test Driven Development''. The repository to be forked is named ``Assignment 5''. You and your group have access to this system using your BGSU ID/Password. The fork must be created under the project ``Assignment \#5 -- Test Driven Development'' so that I have access to it. \\

    

    \newpage
    \textbf{Commands for Branching/Merging.}\\ 

    \begin{tabular}{ l l }
        Create a branch:                    & \texttt{git branch BRANCHNAME}        \\
        Switch to a branch:                 & \texttt{git checkout BRANCHNAME}      \\
        Create/Switch to a branch:          & \texttt{git checkout -b BRANCHNAME}   \\
        Switch to master branch:            & \texttt{git checkout master}          \\
        Merge branch into current branch:   & \texttt{git merge BRANCH\_TO\_MERGE\_INTO\_CURRENT\_BRANCH}          \\
    \end{tabular} \\


    \textbf{Resources for Test-Driven Development}
    \begin{itemize}[noitemsep]
        \item \url{https://docs.python.org/2/library/unittest.html}
        \item \url{http://pymotw.com/2/unittest/}
        \item \url{http://www.jeffknupp.com/blog/2013/12/09/improve-your-python-understanding-unit-testing/}
        \item \url{http://www.drdobbs.com/testing/unit-testing-with-python/240165163}
        \item \url{http://code.tutsplus.com/tutorials/the-newbies-guide-to-test-driven-development--net-13835}

    \end{itemize}

    \textbf{Resources for Branching/Merging with git}
    \begin{itemize}[noitemsep]
        \item \url{http://pcottle.github.io/learnGitBranching/}
        \item \url{https://www.atlassian.com/git/tutorials/using-branches/git-merge}
        \item \url{http://git-scm.com/book/en/v2/Git-Branching-Basic-Branching-and-Merging}
        \item \url{http://rogerdudler.github.io/git-guide/}
        \item \url{http://www.gitguys.com/topics/merging-branches-without-a-conflict/}
        \item \url{http://rypress.com/tutorials/git/branches-2}
        \item \url{http://nvie.com/posts/a-successful-git-branching-model/}
    \end{itemize}


    \textbf{Tutorials for various git tools:}
    \begin{itemize}[noitemsep]
        \item Git GUI
        \begin{itemize}
            \item \url{http://nathanj.github.io/gitguide/tour.html}
        \end{itemize}
        \item SourceTree
        \begin{itemize}
            \item \url{http://rancoud.com/sourcetree-git-use/}. (Help is also built in to the product.)
        \end{itemize}
        \item TortoiseGit
        \begin{itemize}
            \item \url{https://www.youtube.com/watch?v=pp2S2lHjzZI}
            \item \url{http://robertgreiner.com/2010/02/getting-started-with-git-and-tortoisegit-on-windows/}
        \end{itemize}
    \end{itemize}

    
\end{document}